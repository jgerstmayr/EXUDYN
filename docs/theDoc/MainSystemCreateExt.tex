\begin{flushleft}
\noindent {def {\bf \exuUrl{https://github.com/jgerstmayr/EXUDYN/blob/master/main/pythonDev/exudyn/mainSystemExtensions.py\#L136}{CreateGround}{}}}\label{sec:mainsystemextensions:CreateGround}
({\it name}= '', {\it referencePosition}= [0.,0.,0.], {\it referenceRotationMatrix}= np.eye(3), {\it graphicsDataList}= [], {\it graphicsDataUserFunction}= 0, {\it show}= True)
\end{flushleft}
\setlength{\itemindent}{0.7cm}
\begin{itemize}[leftmargin=0.7cm]
\item[--]
{\bf function description}: \vspace{-6pt}
\begin{itemize}[leftmargin=1.2cm]
\setlength{\itemindent}{-0.7cm}
\item[]helper function to create a ground object, using arguments of ObjectGround; this function is mainly added for consistency with other mainSystemExtensions
\item[]- NOTE that this function is added to MainSystem via Python function MainSystemCreateGround.
\end{itemize}
\item[--]
{\bf input}: \vspace{-6pt}
\begin{itemize}[leftmargin=1.2cm]
\setlength{\itemindent}{-0.7cm}
\item[]{\it name}: name string for object
\item[]{\it referencePosition}: reference coordinates for point node (always a 3D vector, no matter if 2D or 3D mass)
\item[]{\it referenceRotationMatrix}: reference rotation matrix for rigid body node (always 3D matrix, no matter if 2D or 3D body)
\item[]{\it graphicsDataList}: list of GraphicsData for optional ground visualization
\item[]{\it graphicsDataUserFunction}: a user function graphicsDataUserFunction(mbs, itemNumber)->BodyGraphicsData (list of GraphicsData), which can be used to draw user-defined graphics; this is much slower than regular GraphicsData
\item[]{\it color}: color of node
\item[]{\it show}: True: show ground object;
\end{itemize}
\item[--]
{\bf output}: ObjectIndex; returns ground object index
\item[--]
{\bf example}: \vspace{-12pt}\ei\begin{lstlisting}[language=Python, xleftmargin=36pt]
  import exudyn as exu
  from exudyn.utilities import * #includes itemInterface and rigidBodyUtilities
  import numpy as np
  SC = exu.SystemContainer()
  mbs = SC.AddSystem()
  ground=mbs.CreateGround(referencePosition = [2,0,0],
                          graphicsDataList = [exu.graphics.CheckerBoard(point=[0,0,0], normal=[0,1,0],size=4)])
\end{lstlisting}\vspace{-24pt}\bi\item[]\vspace{-24pt}\vspace{12pt}\end{itemize}
%

%
\noindent For examples on CreateGround see Relevant Examples (Ex) and TestModels (TM) with weblink to github:
\bi
 \item \footnotesize \exuUrl{https://github.com/jgerstmayr/EXUDYN/blob/master/main/pythonDev/Examples/basicTutorial2024.py}{\texttt{basicTutorial2024.py}} (Ex), 
\exuUrl{https://github.com/jgerstmayr/EXUDYN/blob/master/main/pythonDev/Examples/beamTutorial.py}{\texttt{beamTutorial.py}} (Ex), 
\exuUrl{https://github.com/jgerstmayr/EXUDYN/blob/master/main/pythonDev/Examples/bicycleIftommBenchmark.py}{\texttt{bicycleIftommBenchmark.py}} (Ex), 
\\ \exuUrl{https://github.com/jgerstmayr/EXUDYN/blob/master/main/pythonDev/Examples/bungeeJump.py}{\texttt{bungeeJump.py}} (Ex), 
\exuUrl{https://github.com/jgerstmayr/EXUDYN/blob/master/main/pythonDev/Examples/cartesianSpringDamper.py}{\texttt{cartesianSpringDamper.py}} (Ex), 
 ...
, 
\exuUrl{https://github.com/jgerstmayr/EXUDYN/blob/master/main/pythonDev/TestModels/ConvexContactTest.py}{\texttt{ConvexContactTest.py}} (TM), 
\\ \exuUrl{https://github.com/jgerstmayr/EXUDYN/blob/master/main/pythonDev/TestModels/rigidBodySpringDamperIntrinsic.py}{\texttt{rigidBodySpringDamperIntrinsic.py}} (TM), 
\exuUrl{https://github.com/jgerstmayr/EXUDYN/blob/master/main/pythonDev/TestModels/sliderCrank3Dbenchmark.py}{\texttt{sliderCrank3Dbenchmark.py}} (TM), 
 ...

\ei

%
\begin{flushleft}
\noindent {def {\bf \exuUrl{https://github.com/jgerstmayr/EXUDYN/blob/master/main/pythonDev/exudyn/mainSystemExtensions.py\#L205}{CreateMassPoint}{}}}\label{sec:mainsystemextensions:CreateMassPoint}
({\it name}= '', {\it referencePosition}= [0.,0.,0.], {\it initialDisplacement}= [0.,0.,0.], {\it initialVelocity}= [0.,0.,0.], {\it physicsMass}= 0, {\it gravity}= [0.,0.,0.], {\it graphicsDataList}= [], {\it drawSize}= -1, {\it color}= [-1.,-1.,-1.,-1.], {\it show}= True, {\it create2D}= False, {\it returnDict}= False)
\end{flushleft}
\setlength{\itemindent}{0.7cm}
\begin{itemize}[leftmargin=0.7cm]
\item[--]
{\bf function description}: \vspace{-6pt}
\begin{itemize}[leftmargin=1.2cm]
\setlength{\itemindent}{-0.7cm}
\item[]helper function to create 2D or 3D mass point object and node, using arguments as in NodePoint and MassPoint
\item[]- NOTE that this function is added to MainSystem via Python function MainSystemCreateMassPoint.
\end{itemize}
\item[--]
{\bf input}: \vspace{-6pt}
\begin{itemize}[leftmargin=1.2cm]
\setlength{\itemindent}{-0.7cm}
\item[]{\it name}: name string for object, node is 'Node:'+name
\item[]{\it referencePosition}: reference coordinates for point node (always a 3D vector, no matter if 2D or 3D mass)
\item[]{\it initialDisplacement}: initial displacements for point node (always a 3D vector, no matter if 2D or 3D mass)
\item[]{\it initialVelocity}: initial velocities for point node (always a 3D vector, no matter if 2D or 3D mass)
\item[]{\it physicsMass}: mass of mass point
\item[]{\it gravity}: gravity vevtor applied (always a 3D vector, no matter if 2D or 3D mass)
\item[]{\it graphicsDataList}: list of GraphicsData for optional mass visualization
\item[]{\it drawSize}: general drawing size of node
\item[]{\it color}: color of node
\item[]{\it show}: True: if graphicsData list is empty, node is shown, otherwise body is shown; otherwise, nothing is shown
\item[]{\it create2D}: if True, create NodePoint2D and MassPoint2D
\item[]{\it returnDict}: if False, returns object index; if True, returns dict of all information on created object and node
\end{itemize}
\item[--]
{\bf output}: Union[dict, ObjectIndex]; returns mass point object index or dict with all data on request (if returnDict=True)
\item[--]
{\bf example}: \vspace{-12pt}\ei\begin{lstlisting}[language=Python, xleftmargin=36pt]
  import exudyn as exu
  from exudyn.utilities import * #includes itemInterface and rigidBodyUtilities
  import numpy as np
  SC = exu.SystemContainer()
  mbs = SC.AddSystem()
  b0=mbs.CreateMassPoint(referencePosition = [0,0,0],
                         initialVelocity = [2,5,0],
                         physicsMass = 1, gravity = [0,-9.81,0],
                         drawSize = 0.5, color=exu.graphics.color.blue)
  mbs.Assemble()
  simulationSettings = exu.SimulationSettings() #takes currently set values or default values
  simulationSettings.timeIntegration.numberOfSteps = 1000
  simulationSettings.timeIntegration.endTime = 2
  mbs.SolveDynamic(simulationSettings = simulationSettings)
\end{lstlisting}\vspace{-24pt}\bi\item[]\vspace{-24pt}\vspace{12pt}\end{itemize}
%

%
\noindent For examples on CreateMassPoint see Relevant Examples (Ex) and TestModels (TM) with weblink to github:
\bi
 \item \footnotesize \exuUrl{https://github.com/jgerstmayr/EXUDYN/blob/master/main/pythonDev/Examples/basicTutorial2024.py}{\texttt{basicTutorial2024.py}} (Ex), 
\exuUrl{https://github.com/jgerstmayr/EXUDYN/blob/master/main/pythonDev/Examples/cartesianSpringDamper.py}{\texttt{cartesianSpringDamper.py}} (Ex), 
\exuUrl{https://github.com/jgerstmayr/EXUDYN/blob/master/main/pythonDev/Examples/cartesianSpringDamperUserFunction.py}{\texttt{cartesianSpringDamperUserFunction.py}} (Ex), 
\\ \exuUrl{https://github.com/jgerstmayr/EXUDYN/blob/master/main/pythonDev/Examples/chatGPTupdate.py}{\texttt{chatGPTupdate.py}} (Ex), 
\exuUrl{https://github.com/jgerstmayr/EXUDYN/blob/master/main/pythonDev/Examples/springDamperTutorialNew.py}{\texttt{springDamperTutorialNew.py}} (Ex), 
 ...
, 
\exuUrl{https://github.com/jgerstmayr/EXUDYN/blob/master/main/pythonDev/TestModels/mainSystemExtensionsTests.py}{\texttt{mainSystemExtensionsTests.py}} (TM), 
\\ \exuUrl{https://github.com/jgerstmayr/EXUDYN/blob/master/main/pythonDev/TestModels/symbolicUserFunctionTest.py}{\texttt{symbolicUserFunctionTest.py}} (TM), 
\exuUrl{https://github.com/jgerstmayr/EXUDYN/blob/master/main/pythonDev/TestModels/taskmanagerTest.py}{\texttt{taskmanagerTest.py}} (TM), 
 ...

\ei

%
\begin{flushleft}
\noindent {def {\bf \exuUrl{https://github.com/jgerstmayr/EXUDYN/blob/master/main/pythonDev/exudyn/mainSystemExtensions.py\#L336}{CreateRigidBody}{}}}\label{sec:mainsystemextensions:CreateRigidBody}
({\it name}= '', {\it referencePosition}= [0.,0.,0.], {\it referenceRotationMatrix}= np.eye(3), {\it initialVelocity}= [0.,0.,0.], {\it initialAngularVelocity}= [0.,0.,0.], {\it initialDisplacement}= None, {\it initialRotationMatrix}= None, {\it inertia}= None, {\it gravity}= [0.,0.,0.], {\it nodeType}= exudyn.NodeType.RotationEulerParameters, {\it graphicsDataList}= [], {\it graphicsDataUserFunction}= 0, {\it drawSize}= -1, {\it color}= [-1.,-1.,-1.,-1.], {\it show}= True, {\it create2D}= False, {\it returnDict}= False)
\end{flushleft}
\setlength{\itemindent}{0.7cm}
\begin{itemize}[leftmargin=0.7cm]
\item[--]
{\bf function description}: \vspace{-6pt}
\begin{itemize}[leftmargin=1.2cm]
\setlength{\itemindent}{-0.7cm}
\item[]helper function to create 3D (or 2D) rigid body object and node; all quantities are global (angular velocity, etc.)
\item[]- NOTE that this function is added to MainSystem via Python function MainSystemCreateRigidBody.
\end{itemize}
\item[--]
{\bf input}: \vspace{-6pt}
\begin{itemize}[leftmargin=1.2cm]
\setlength{\itemindent}{-0.7cm}
\item[]{\it name}: name string for object, node is 'Node:'+name
\item[]{\it referencePosition}: reference position vector for rigid body node (always a 3D vector, no matter if 2D or 3D body)
\item[]{\it referenceRotationMatrix}: reference rotation matrix for rigid body node (always 3D matrix, no matter if 2D or 3D body)
\item[]{\it initialVelocity}: initial translational velocity vector for node (always a 3D vector, no matter if 2D or 3D body)
\item[]{\it initialAngularVelocity}: initial angular velocity vector for node (always a 3D vector, no matter if 2D or 3D body)
\item[]{\it initialDisplacement}: initial translational displacement vector for node (always a 3D vector, no matter if 2D or 3D body); these displacements are deviations from reference position, e.g. for a finite element node [None: unused]
\item[]{\it initialRotationMatrix}: initial rotation provided as matrix (always a 3D matrix, no matter if 2D or 3D body); this rotation is superimposed to reference rotation [None: unused]
\item[]{\it inertia}: an instance of class RigidBodyInertia, see rigidBodyUtilities; may also be from derived class (InertiaCuboid, InertiaMassPoint, InertiaCylinder, ...)
\item[]{\it gravity}: gravity vevtor applied (always a 3D vector, no matter if 2D or 3D mass)
\item[]{\it graphicsDataList}: list of GraphicsData for rigid body visualization; use exudyn.graphics functions to create GraphicsData for basic solids
\item[]{\it graphicsDataUserFunction}: a user function graphicsDataUserFunction(mbs, itemNumber)->BodyGraphicsData (list of GraphicsData), which can be used to draw user-defined graphics; this is much slower than regular GraphicsData
\item[]{\it drawSize}: general drawing size of node
\item[]{\it color}: color of node
\item[]{\it show}: True: if graphicsData list is empty, node is shown, otherwise body is shown; False: nothing is shown
\item[]{\it create2D}: if True, create NodeRigidBody2D and ObjectRigidBody2D
\item[]{\it returnDict}: if False, returns object index; if True, returns dict of all information on created object and node
\end{itemize}
\item[--]
{\bf output}: Union[dict, ObjectIndex]; returns rigid body object index (or dict with 'nodeNumber', 'objectNumber' and possibly 'loadNumber' and 'markerBodyMass' if returnDict=True)
\item[--]
{\bf example}: \vspace{-12pt}\ei\begin{lstlisting}[language=Python, xleftmargin=36pt]
  import exudyn as exu
  from exudyn.utilities import * #includes itemInterface and rigidBodyUtilities
  import numpy as np
  SC = exu.SystemContainer()
  mbs = SC.AddSystem()
  b0 = mbs.CreateRigidBody(inertia = InertiaCuboid(density=5000,
                                                   sideLengths=[1,0.1,0.1]),
                           referencePosition = [1,0,0],
                           initialVelocity = [2,5,0],
                           initialAngularVelocity = [5,0.5,0.7],
                           gravity = [0,-9.81,0],
                           graphicsDataList = [exu.graphics.Brick(size=[1,0.1,0.1],
                                                                        color=exu.graphics.color.red)])
  mbs.Assemble()
  simulationSettings = exu.SimulationSettings() #takes currently set values or default values
  simulationSettings.timeIntegration.numberOfSteps = 1000
  simulationSettings.timeIntegration.endTime = 2
  mbs.SolveDynamic(simulationSettings = simulationSettings)
\end{lstlisting}\vspace{-24pt}\bi\item[]\vspace{-24pt}\vspace{12pt}\end{itemize}
%

%
\noindent For examples on CreateRigidBody see Relevant Examples (Ex) and TestModels (TM) with weblink to github:
\bi
 \item \footnotesize \exuUrl{https://github.com/jgerstmayr/EXUDYN/blob/master/main/pythonDev/Examples/addPrismaticJoint.py}{\texttt{addPrismaticJoint.py}} (Ex), 
\exuUrl{https://github.com/jgerstmayr/EXUDYN/blob/master/main/pythonDev/Examples/addRevoluteJoint.py}{\texttt{addRevoluteJoint.py}} (Ex), 
\exuUrl{https://github.com/jgerstmayr/EXUDYN/blob/master/main/pythonDev/Examples/ANCFrotatingCable2D.py}{\texttt{ANCFrotatingCable2D.py}} (Ex), 
\\ \exuUrl{https://github.com/jgerstmayr/EXUDYN/blob/master/main/pythonDev/Examples/bicycleIftommBenchmark.py}{\texttt{bicycleIftommBenchmark.py}} (Ex), 
\exuUrl{https://github.com/jgerstmayr/EXUDYN/blob/master/main/pythonDev/Examples/bungeeJump.py}{\texttt{bungeeJump.py}} (Ex), 
 ...
, 
\exuUrl{https://github.com/jgerstmayr/EXUDYN/blob/master/main/pythonDev/TestModels/bricardMechanism.py}{\texttt{bricardMechanism.py}} (TM), 
\\ \exuUrl{https://github.com/jgerstmayr/EXUDYN/blob/master/main/pythonDev/TestModels/carRollingDiscTest.py}{\texttt{carRollingDiscTest.py}} (TM), 
\exuUrl{https://github.com/jgerstmayr/EXUDYN/blob/master/main/pythonDev/TestModels/complexEigenvaluesTest.py}{\texttt{complexEigenvaluesTest.py}} (TM), 
 ...

\ei

%
\begin{flushleft}
\noindent {def {\bf \exuUrl{https://github.com/jgerstmayr/EXUDYN/blob/master/main/pythonDev/exudyn/mainSystemExtensions.py\#L564}{CreateSpringDamper}{}}}\label{sec:mainsystemextensions:CreateSpringDamper}
({\it name}= '', {\it bodyNumbers}= [None, None], {\it localPosition0}= [0.,0.,0.], {\it localPosition1}= [0.,0.,0.], {\it referenceLength}= None, {\it stiffness}= 0., {\it damping}= 0., {\it force}= 0., {\it velocityOffset}= 0., {\it springForceUserFunction}= 0, {\it bodyOrNodeList}= [None, None], {\it bodyList}= [None, None], {\it show}= True, {\it drawSize}= -1, {\it color}= exudyn.graphics.color.default)
\end{flushleft}
\setlength{\itemindent}{0.7cm}
\begin{itemize}[leftmargin=0.7cm]
\item[--]
{\bf function description}: \vspace{-6pt}
\begin{itemize}[leftmargin=1.2cm]
\setlength{\itemindent}{-0.7cm}
\item[]helper function to create SpringDamper connector, using arguments from ObjectConnectorSpringDamper; similar interface as CreateDistanceConstraint(...), see there for for further information
\item[]- NOTE that this function is added to MainSystem via Python function MainSystemCreateSpringDamper.
\end{itemize}
\item[--]
{\bf input}: \vspace{-6pt}
\begin{itemize}[leftmargin=1.2cm]
\setlength{\itemindent}{-0.7cm}
\item[]{\it name}: name string for connector; markers get Marker0:name and Marker1:name
\item[]{\it bodyNumbers}: a list of two body numbers (ObjectIndex) to be connected
\item[]{\it localPosition0}: local position (as 3D list or numpy array) on body0, if not a node number
\item[]{\it localPosition1}: local position (as 3D list or numpy array) on body1, if not a node number
\item[]{\it referenceLength}: if None, length is computed from reference position of bodies or nodes; if not None, this scalar reference length is used for spring
\item[]{\it stiffness}: scalar stiffness coefficient
\item[]{\it damping}: scalar damping coefficient
\item[]{\it force}: scalar additional force applied
\item[]{\it velocityOffset}: scalar offset: if referenceLength is changed over time, the velocityOffset may be changed accordingly to emulate a reference motion
\item[]{\it springForceUserFunction}: a user function springForceUserFunction(mbs, t, itemNumber, deltaL, deltaL\_t, stiffness, damping, force)->float ; this function replaces the internal connector force compuation
\item[]{\it bodyOrNodeList}: alternative to bodyNumbers; a list of object numbers (with specific localPosition0/1) or node numbers; may alse be mixed types; to use this case, set bodyNumbers = [None,None]
\item[]{\it show}: if True, connector visualization is drawn
\item[]{\it drawSize}: general drawing size of connector
\item[]{\it color}: color of connector
\end{itemize}
\item[--]
{\bf output}: ObjectIndex; returns index of newly created object
\item[--]
{\bf example}: \vspace{-12pt}\ei\begin{lstlisting}[language=Python, xleftmargin=36pt]
  import exudyn as exu
  from exudyn.utilities import * #includes itemInterface and rigidBodyUtilities
  import numpy as np
  SC = exu.SystemContainer()
  mbs = SC.AddSystem()
  b0 = mbs.CreateMassPoint(referencePosition = [2,0,0],
                           initialVelocity = [2,5,0],
                           physicsMass = 1, gravity = [0,-9.81,0],
                           drawSize = 0.5, color=exu.graphics.color.blue)
  oGround = mbs.AddObject(ObjectGround())
  #add vertical spring
  oSD = mbs.CreateSpringDamper(bodyNumbers=[oGround, b0],
                               localPosition0=[2,1,0],
                               localPosition1=[0,0,0],
                               stiffness=1e4, damping=1e2,
                               drawSize=0.2)
  mbs.Assemble()
  simulationSettings = exu.SimulationSettings() #takes currently set values or default values
  simulationSettings.timeIntegration.numberOfSteps = 1000
  simulationSettings.timeIntegration.endTime = 2
  SC.visualizationSettings.nodes.drawNodesAsPoint=False
  mbs.SolveDynamic(simulationSettings = simulationSettings)
\end{lstlisting}\vspace{-24pt}\bi\item[]\vspace{-24pt}\vspace{12pt}\end{itemize}
%

%
\noindent For examples on CreateSpringDamper see Relevant Examples (Ex) and TestModels (TM) with weblink to github:
\bi
 \item \footnotesize \exuUrl{https://github.com/jgerstmayr/EXUDYN/blob/master/main/pythonDev/Examples/basicTutorial2024.py}{\texttt{basicTutorial2024.py}} (Ex), 
\exuUrl{https://github.com/jgerstmayr/EXUDYN/blob/master/main/pythonDev/Examples/chatGPTupdate.py}{\texttt{chatGPTupdate.py}} (Ex), 
\exuUrl{https://github.com/jgerstmayr/EXUDYN/blob/master/main/pythonDev/Examples/springDamperTutorialNew.py}{\texttt{springDamperTutorialNew.py}} (Ex), 
\\ \exuUrl{https://github.com/jgerstmayr/EXUDYN/blob/master/main/pythonDev/Examples/springMassFriction.py}{\texttt{springMassFriction.py}} (Ex), 
\exuUrl{https://github.com/jgerstmayr/EXUDYN/blob/master/main/pythonDev/Examples/symbolicUserFunctionMasses.py}{\texttt{symbolicUserFunctionMasses.py}} (Ex), 
 ...
, 
\exuUrl{https://github.com/jgerstmayr/EXUDYN/blob/master/main/pythonDev/TestModels/mainSystemExtensionsTests.py}{\texttt{mainSystemExtensionsTests.py}} (TM), 
\\ \exuUrl{https://github.com/jgerstmayr/EXUDYN/blob/master/main/pythonDev/TestModels/symbolicUserFunctionTest.py}{\texttt{symbolicUserFunctionTest.py}} (TM), 
\exuUrl{https://github.com/jgerstmayr/EXUDYN/blob/master/main/pythonDev/TestModels/taskmanagerTest.py}{\texttt{taskmanagerTest.py}} (TM), 
 ...

\ei

%
\begin{flushleft}
\noindent {def {\bf \exuUrl{https://github.com/jgerstmayr/EXUDYN/blob/master/main/pythonDev/exudyn/mainSystemExtensions.py\#L698}{CreateCartesianSpringDamper}{}}}\label{sec:mainsystemextensions:CreateCartesianSpringDamper}
({\it name}= '', {\it bodyNumbers}= [None, None], {\it localPosition0}= [0.,0.,0.], {\it localPosition1}= [0.,0.,0.], {\it stiffness}= [0.,0.,0.], {\it damping}= [0.,0.,0.], {\it offset}= [0.,0.,0.], {\it springForceUserFunction}= 0, {\it bodyOrNodeList}= [None, None], {\it bodyList}= [None, None], {\it show}= True, {\it drawSize}= -1, {\it color}= exudyn.graphics.color.default)
\end{flushleft}
\setlength{\itemindent}{0.7cm}
\begin{itemize}[leftmargin=0.7cm]
\item[--]
{\bf function description}: \vspace{-6pt}
\begin{itemize}[leftmargin=1.2cm]
\setlength{\itemindent}{-0.7cm}
\item[]helper function to create CartesianSpringDamper connector, using arguments from ObjectConnectorCartesianSpringDamper
\item[]- NOTE that this function is added to MainSystem via Python function MainSystemCreateCartesianSpringDamper.
\end{itemize}
\item[--]
{\bf input}: \vspace{-6pt}
\begin{itemize}[leftmargin=1.2cm]
\setlength{\itemindent}{-0.7cm}
\item[]{\it name}: name string for connector; markers get Marker0:name and Marker1:name
\item[]{\it bodyNumbers}: a list of two body numbers (ObjectIndex) to be connected
\item[]{\it localPosition0}: local position (as 3D list or numpy array) on body0, if not a node number
\item[]{\it localPosition1}: local position (as 3D list or numpy array) on body1, if not a node number
\item[]{\it stiffness}: stiffness coefficients (as 3D list or numpy array)
\item[]{\it damping}: damping coefficients (as 3D list or numpy array)
\item[]{\it offset}: offset vector (as 3D list or numpy array)
\item[]{\it springForceUserFunction}: a user function springForceUserFunction(mbs, t, itemNumber, displacement, velocity, stiffness, damping, offset)->[float,float,float] ; this function replaces the internal connector force compuation
\item[]{\it bodyOrNodeList}: alternative to bodyNumbers; a list of object numbers (with specific localPosition0/1) or node numbers; may alse be mixed types; to use this case, set bodyNumbers = [None,None]
\item[]{\it show}: if True, connector visualization is drawn
\item[]{\it drawSize}: general drawing size of connector
\item[]{\it color}: color of connector
\end{itemize}
\item[--]
{\bf output}: ObjectIndex; returns index of newly created object
\item[--]
{\bf example}: \vspace{-12pt}\ei\begin{lstlisting}[language=Python, xleftmargin=36pt]
  import exudyn as exu
  from exudyn.utilities import * #includes itemInterface and rigidBodyUtilities
  import numpy as np
  SC = exu.SystemContainer()
  mbs = SC.AddSystem()
  b0 = mbs.CreateMassPoint(referencePosition = [7,0,0],
                            physicsMass = 1, gravity = [0,-9.81,0],
                            drawSize = 0.5, color=exu.graphics.color.blue)
  oGround = mbs.AddObject(ObjectGround())
  oSD = mbs.CreateCartesianSpringDamper(bodyNumbers=[oGround, b0],
                                localPosition0=[7.5,1,0],
                                localPosition1=[0,0,0],
                                stiffness=[200,2000,0], damping=[2,20,0],
                                drawSize=0.2)
  mbs.Assemble()
  simulationSettings = exu.SimulationSettings() #takes currently set values or default values
  simulationSettings.timeIntegration.numberOfSteps = 1000
  simulationSettings.timeIntegration.endTime = 2
  SC.visualizationSettings.nodes.drawNodesAsPoint=False
  mbs.SolveDynamic(simulationSettings = simulationSettings)
\end{lstlisting}\vspace{-24pt}\bi\item[]\vspace{-24pt}\vspace{12pt}\end{itemize}
%

%
\noindent For examples on CreateCartesianSpringDamper see Relevant Examples (Ex) and TestModels (TM) with weblink to github:
\bi
 \item \footnotesize \exuUrl{https://github.com/jgerstmayr/EXUDYN/blob/master/main/pythonDev/Examples/cartesianSpringDamper.py}{\texttt{cartesianSpringDamper.py}} (Ex), 
\exuUrl{https://github.com/jgerstmayr/EXUDYN/blob/master/main/pythonDev/Examples/cartesianSpringDamperUserFunction.py}{\texttt{cartesianSpringDamperUserFunction.py}} (Ex), 
\exuUrl{https://github.com/jgerstmayr/EXUDYN/blob/master/main/pythonDev/Examples/chatGPTupdate.py}{\texttt{chatGPTupdate.py}} (Ex), 
\\ \exuUrl{https://github.com/jgerstmayr/EXUDYN/blob/master/main/pythonDev/TestModels/complexEigenvaluesTest.py}{\texttt{complexEigenvaluesTest.py}} (TM), 
\exuUrl{https://github.com/jgerstmayr/EXUDYN/blob/master/main/pythonDev/TestModels/computeODE2AEeigenvaluesTest.py}{\texttt{computeODE2AEeigenvaluesTest.py}} (TM), 
\exuUrl{https://github.com/jgerstmayr/EXUDYN/blob/master/main/pythonDev/TestModels/mainSystemExtensionsTests.py}{\texttt{mainSystemExtensionsTests.py}} (TM)
\ei

%
\begin{flushleft}
\noindent {def {\bf \exuUrl{https://github.com/jgerstmayr/EXUDYN/blob/master/main/pythonDev/exudyn/mainSystemExtensions.py\#L787}{CreateRigidBodySpringDamper}{}}}\label{sec:mainsystemextensions:CreateRigidBodySpringDamper}
({\it name}= '', {\it bodyNumbers}= [None, None], {\it localPosition0}= [0.,0.,0.], {\it localPosition1}= [0.,0.,0.], {\it stiffness}= np.zeros((6,6)), {\it damping}= np.zeros((6,6)), {\it offset}= [0.,0.,0.,0.,0.,0.], {\it rotationMatrixJoint}= np.eye(3), {\it useGlobalFrame}= True, {\it intrinsicFormulation}= True, {\it springForceTorqueUserFunction}= 0, {\it postNewtonStepUserFunction}= 0, {\it bodyOrNodeList}= [None, None], {\it bodyList}= [None, None], {\it show}= True, {\it drawSize}= -1, {\it color}= exudyn.graphics.color.default)
\end{flushleft}
\setlength{\itemindent}{0.7cm}
\begin{itemize}[leftmargin=0.7cm]
\item[--]
{\bf function description}: \vspace{-6pt}
\begin{itemize}[leftmargin=1.2cm]
\setlength{\itemindent}{-0.7cm}
\item[]helper function to create RigidBodySpringDamper connector, using arguments from ObjectConnectorRigidBodySpringDamper, see there for the full documentation
\item[]- NOTE that this function is added to MainSystem via Python function MainSystemCreateRigidBodySpringDamper.
\end{itemize}
\item[--]
{\bf input}: \vspace{-6pt}
\begin{itemize}[leftmargin=1.2cm]
\setlength{\itemindent}{-0.7cm}
\item[]{\it name}: name string for connector; markers get Marker0:name and Marker1:name
\item[]{\it bodyNumbers}: a list of two body numbers (ObjectIndex) to be connected
\item[]{\it localPosition0}: local position (as 3D list or numpy array) on body0, if not a node number
\item[]{\it localPosition1}: local position (as 3D list or numpy array) on body1, if not a node number
\item[]{\it stiffness}: stiffness coefficients (as 6D matrix or numpy array)
\item[]{\it damping}: damping coefficients (as 6D matrix or numpy array)
\item[]{\it offset}: offset vector (as 6D list or numpy array)
\item[]{\it rotationMatrixJoint}: additional rotation matrix; in case  useGlobalFrame=False, it transforms body0/node0 local frame to joint frame; if useGlobalFrame=True, it transforms global frame to joint frame
\item[]{\it useGlobalFrame}: if False, the rotationMatrixJoint is defined in the local coordinate system of body0
\item[]{\it intrinsicFormulation}: if True, uses intrinsic formulation of Maserati and Morandini, which uses matrix logarithm and is independent of order of markers (preferred formulation); otherwise, Tait-Bryan angles are used for computation of torque, see documentation
\item[]{\it springForceTorqueUserFunction}: a user function springForceTorqueUserFunction(mbs, t, itemNumber, displacement, rotation, velocity, angularVelocity, stiffness, damping, rotJ0, rotJ1, offset)->[float,float,float, float,float,float] ; this function replaces the internal connector force / torque compuation
\item[]{\it postNewtonStepUserFunction}: a special user function postNewtonStepUserFunction(mbs, t, Index itemIndex, dataCoordinates, displacement, rotation, velocity, angularVelocity, stiffness, damping, rotJ0, rotJ1, offset)->[PNerror, recommendedStepSize, data[0], data[1], ...] ; for details, see RigidBodySpringDamper for full docu
\item[]{\it bodyOrNodeList}: alternative to bodyNumbers; a list of object numbers (with specific localPosition0/1) or node numbers; may alse be mixed types; to use this case, set bodyNumbers = [None,None]
\item[]{\it show}: if True, connector visualization is drawn
\item[]{\it drawSize}: general drawing size of connector
\item[]{\it color}: color of connector
\end{itemize}
\item[--]
{\bf output}: ObjectIndex; returns index of newly created object
\item[--]
{\bf example}: \vspace{-12pt}\ei\begin{lstlisting}[language=Python, xleftmargin=36pt]
  #coming later
\end{lstlisting}\vspace{-24pt}\bi\item[]\vspace{-24pt}\vspace{12pt}\end{itemize}
%

%
\noindent For examples on CreateRigidBodySpringDamper see Relevant Examples (Ex) and TestModels (TM) with weblink to github:
\bi
 \item \footnotesize \exuUrl{https://github.com/jgerstmayr/EXUDYN/blob/master/main/pythonDev/TestModels/bricardMechanism.py}{\texttt{bricardMechanism.py}} (TM), 
\exuUrl{https://github.com/jgerstmayr/EXUDYN/blob/master/main/pythonDev/TestModels/rigidBodySpringDamperIntrinsic.py}{\texttt{rigidBodySpringDamperIntrinsic.py}} (TM)
\ei

%
\begin{flushleft}
\noindent {def {\bf \exuUrl{https://github.com/jgerstmayr/EXUDYN/blob/master/main/pythonDev/exudyn/mainSystemExtensions.py\#L933}{CreateRevoluteJoint}{}}}\label{sec:mainsystemextensions:CreateRevoluteJoint}
({\it name}= '', {\it bodyNumbers}= [None, None], {\it position}= [], {\it axis}= [], {\it useGlobalFrame}= True, {\it show}= True, {\it axisRadius}= 0.1, {\it axisLength}= 0.4, {\it color}= exudyn.graphics.color.default)
\end{flushleft}
\setlength{\itemindent}{0.7cm}
\begin{itemize}[leftmargin=0.7cm]
\item[--]
{\bf function description}: \vspace{-6pt}
\begin{itemize}[leftmargin=1.2cm]
\setlength{\itemindent}{-0.7cm}
\item[]Create revolute joint between two bodies; definition of joint position and axis in global coordinates (alternatively in body0 local coordinates) for reference configuration of bodies; all markers, markerRotation and other quantities are automatically computed
\item[]- NOTE that this function is added to MainSystem via Python function MainSystemCreateRevoluteJoint.
\end{itemize}
\item[--]
{\bf input}: \vspace{-6pt}
\begin{itemize}[leftmargin=1.2cm]
\setlength{\itemindent}{-0.7cm}
\item[]{\it name}: name string for joint; markers get Marker0:name and Marker1:name
\item[]{\it bodyNumbers}: a list of object numbers for body0 and body1; must be rigid body or ground object
\item[]{\it position}: a 3D vector as list or np.array: if useGlobalFrame=True it describes the global position of the joint in reference configuration; else: local position in body0
\item[]{\it axis}: a 3D vector as list or np.array containing the joint axis either in local body0 coordinates (useGlobalFrame=False), or in global reference configuration (useGlobalFrame=True)
\item[]{\it useGlobalFrame}: if False, the position and axis vectors are defined in the local coordinate system of body0, otherwise in global (reference) coordinates
\item[]{\it show}: if True, connector visualization is drawn
\item[]{\it axisRadius}: radius of axis for connector graphical representation
\item[]{\it axisLength}: length of axis for connector graphical representation
\item[]{\it color}: color of connector
\end{itemize}
\item[--]
{\bf output}: [ObjectIndex, MarkerIndex, MarkerIndex]; returns list [oJoint, mBody0, mBody1], containing the joint object number, and the two rigid body markers on body0/1 for the joint
\item[--]
{\bf example}: \vspace{-12pt}\ei\begin{lstlisting}[language=Python, xleftmargin=36pt]
  import exudyn as exu
  from exudyn.utilities import * #includes itemInterface and rigidBodyUtilities
  import numpy as np
  SC = exu.SystemContainer()
  mbs = SC.AddSystem()
  b0 = mbs.CreateRigidBody(inertia = InertiaCuboid(density=5000,
                                                   sideLengths=[1,0.1,0.1]),
                           referencePosition = [3,0,0],
                           gravity = [0,-9.81,0],
                           graphicsDataList = [exu.graphics.Brick(size=[1,0.1,0.1],
                                                                        color=exu.graphics.color.steelblue)])
  oGround = mbs.AddObject(ObjectGround())
  mbs.CreateRevoluteJoint(bodyNumbers=[oGround, b0], position=[2.5,0,0], axis=[0,0,1],
                          useGlobalFrame=True, axisRadius=0.02, axisLength=0.14)
  mbs.Assemble()
  simulationSettings = exu.SimulationSettings() #takes currently set values or default values
  simulationSettings.timeIntegration.numberOfSteps = 1000
  simulationSettings.timeIntegration.endTime = 2
  mbs.SolveDynamic(simulationSettings = simulationSettings)
\end{lstlisting}\vspace{-24pt}\bi\item[]\vspace{-24pt}\vspace{12pt}\end{itemize}
%

%
\noindent For examples on CreateRevoluteJoint see Relevant Examples (Ex) and TestModels (TM) with weblink to github:
\bi
 \item \footnotesize \exuUrl{https://github.com/jgerstmayr/EXUDYN/blob/master/main/pythonDev/Examples/addRevoluteJoint.py}{\texttt{addRevoluteJoint.py}} (Ex), 
\exuUrl{https://github.com/jgerstmayr/EXUDYN/blob/master/main/pythonDev/Examples/bicycleIftommBenchmark.py}{\texttt{bicycleIftommBenchmark.py}} (Ex), 
\exuUrl{https://github.com/jgerstmayr/EXUDYN/blob/master/main/pythonDev/Examples/chatGPTupdate.py}{\texttt{chatGPTupdate.py}} (Ex), 
\\ \exuUrl{https://github.com/jgerstmayr/EXUDYN/blob/master/main/pythonDev/Examples/chatGPTupdate2.py}{\texttt{chatGPTupdate2.py}} (Ex), 
\exuUrl{https://github.com/jgerstmayr/EXUDYN/blob/master/main/pythonDev/Examples/multiMbsTest.py}{\texttt{multiMbsTest.py}} (Ex), 
 ...
, 
\exuUrl{https://github.com/jgerstmayr/EXUDYN/blob/master/main/pythonDev/TestModels/bricardMechanism.py}{\texttt{bricardMechanism.py}} (TM), 
\\ \exuUrl{https://github.com/jgerstmayr/EXUDYN/blob/master/main/pythonDev/TestModels/mainSystemExtensionsTests.py}{\texttt{mainSystemExtensionsTests.py}} (TM), 
\exuUrl{https://github.com/jgerstmayr/EXUDYN/blob/master/main/pythonDev/TestModels/perf3DRigidBodies.py}{\texttt{perf3DRigidBodies.py}} (TM), 
 ...

\ei

%
\begin{flushleft}
\noindent {def {\bf \exuUrl{https://github.com/jgerstmayr/EXUDYN/blob/master/main/pythonDev/exudyn/mainSystemExtensions.py\#L1035}{CreatePrismaticJoint}{}}}\label{sec:mainsystemextensions:CreatePrismaticJoint}
({\it name}= '', {\it bodyNumbers}= [None, None], {\it position}= [], {\it axis}= [], {\it useGlobalFrame}= True, {\it show}= True, {\it axisRadius}= 0.1, {\it axisLength}= 0.4, {\it color}= exudyn.graphics.color.default)
\end{flushleft}
\setlength{\itemindent}{0.7cm}
\begin{itemize}[leftmargin=0.7cm]
\item[--]
{\bf function description}: \vspace{-6pt}
\begin{itemize}[leftmargin=1.2cm]
\setlength{\itemindent}{-0.7cm}
\item[]Create prismatic joint between two bodies; definition of joint position and axis in global coordinates (alternatively in body0 local coordinates) for reference configuration of bodies; all markers, markerRotation and other quantities are automatically computed
\item[]- NOTE that this function is added to MainSystem via Python function MainSystemCreatePrismaticJoint.
\end{itemize}
\item[--]
{\bf input}: \vspace{-6pt}
\begin{itemize}[leftmargin=1.2cm]
\setlength{\itemindent}{-0.7cm}
\item[]{\it name}: name string for joint; markers get Marker0:name and Marker1:name
\item[]{\it bodyNumbers}: a list of object numbers for body0 and body1; must be rigid body or ground object
\item[]{\it position}: a 3D vector as list or np.array: if useGlobalFrame=True it describes the global position of the joint in reference configuration; else: local position in body0
\item[]{\it axis}: a 3D vector as list or np.array containing the joint axis either in local body0 coordinates (useGlobalFrame=False), or in global reference configuration (useGlobalFrame=True)
\item[]{\it useGlobalFrame}: if False, the position and axis vectors are defined in the local coordinate system of body0, otherwise in global (reference) coordinates
\item[]{\it show}: if True, connector visualization is drawn
\item[]{\it axisRadius}: radius of axis for connector graphical representation
\item[]{\it axisLength}: length of axis for connector graphical representation
\item[]{\it color}: color of connector
\end{itemize}
\item[--]
{\bf output}: [ObjectIndex, MarkerIndex, MarkerIndex]; returns list [oJoint, mBody0, mBody1], containing the joint object number, and the two rigid body markers on body0/1 for the joint
\item[--]
{\bf example}: \vspace{-12pt}\ei\begin{lstlisting}[language=Python, xleftmargin=36pt]
  import exudyn as exu
  from exudyn.utilities import * #includes itemInterface and rigidBodyUtilities
  import numpy as np
  SC = exu.SystemContainer()
  mbs = SC.AddSystem()
  b0 = mbs.CreateRigidBody(inertia = InertiaCuboid(density=5000,
                                                   sideLengths=[1,0.1,0.1]),
                           referencePosition = [4,0,0],
                           initialVelocity = [0,4,0],
                           gravity = [0,-9.81,0],
                           graphicsDataList = [exu.graphics.Brick(size=[1,0.1,0.1],
                                                                        color=exu.graphics.color.steelblue)])
  oGround = mbs.AddObject(ObjectGround())
  mbs.CreatePrismaticJoint(bodyNumbers=[oGround, b0], position=[3.5,0,0], axis=[0,1,0],
                           useGlobalFrame=True, axisRadius=0.02, axisLength=1)
  mbs.Assemble()
  simulationSettings = exu.SimulationSettings() #takes currently set values or default values
  simulationSettings.timeIntegration.numberOfSteps = 1000
  simulationSettings.timeIntegration.endTime = 2
  mbs.SolveDynamic(simulationSettings = simulationSettings)
\end{lstlisting}\vspace{-24pt}\bi\item[]\vspace{-24pt}\vspace{12pt}\end{itemize}
%

%
\noindent For examples on CreatePrismaticJoint see Relevant Examples (Ex) and TestModels (TM) with weblink to github:
\bi
 \item \footnotesize \exuUrl{https://github.com/jgerstmayr/EXUDYN/blob/master/main/pythonDev/Examples/addPrismaticJoint.py}{\texttt{addPrismaticJoint.py}} (Ex), 
\exuUrl{https://github.com/jgerstmayr/EXUDYN/blob/master/main/pythonDev/Examples/chatGPTupdate.py}{\texttt{chatGPTupdate.py}} (Ex), 
\exuUrl{https://github.com/jgerstmayr/EXUDYN/blob/master/main/pythonDev/Examples/chatGPTupdate2.py}{\texttt{chatGPTupdate2.py}} (Ex), 
\\ \exuUrl{https://github.com/jgerstmayr/EXUDYN/blob/master/main/pythonDev/TestModels/mainSystemExtensionsTests.py}{\texttt{mainSystemExtensionsTests.py}} (TM)
\ei

%
\begin{flushleft}
\noindent {def {\bf \exuUrl{https://github.com/jgerstmayr/EXUDYN/blob/master/main/pythonDev/exudyn/mainSystemExtensions.py\#L1129}{CreateSphericalJoint}{}}}\label{sec:mainsystemextensions:CreateSphericalJoint}
({\it name}= '', {\it bodyNumbers}= [None, None], {\it position}= [], {\it constrainedAxes}= [1,1,1], {\it useGlobalFrame}= True, {\it show}= True, {\it jointRadius}= 0.1, {\it color}= exudyn.graphics.color.default)
\end{flushleft}
\setlength{\itemindent}{0.7cm}
\begin{itemize}[leftmargin=0.7cm]
\item[--]
{\bf function description}: \vspace{-6pt}
\begin{itemize}[leftmargin=1.2cm]
\setlength{\itemindent}{-0.7cm}
\item[]Create spherical joint between two bodies; definition of joint position in global coordinates (alternatively in body0 local coordinates) for reference configuration of bodies; all markers are automatically computed
\item[]- NOTE that this function is added to MainSystem via Python function MainSystemCreateSphericalJoint.
\end{itemize}
\item[--]
{\bf input}: \vspace{-6pt}
\begin{itemize}[leftmargin=1.2cm]
\setlength{\itemindent}{-0.7cm}
\item[]{\it name}: name string for joint; markers get Marker0:name and Marker1:name
\item[]{\it bodyNumbers}: a list of object numbers for body0 and body1; must be mass point, rigid body or ground object
\item[]{\it position}: a 3D vector as list or np.array: if useGlobalFrame=True it describes the global position of the joint in reference configuration; else: local position in body0
\item[]{\it constrainedAxes}: flags, which determines which (global) translation axes are constrained; each entry may only be 0 (=free) axis or 1 (=constrained axis)
\item[]{\it useGlobalFrame}: if False, the point and axis vectors are defined in the local coordinate system of body0
\item[]{\it show}: if True, connector visualization is drawn
\item[]{\it jointRadius}: radius of sphere for connector graphical representation
\item[]{\it color}: color of connector
\end{itemize}
\item[--]
{\bf output}: [ObjectIndex, MarkerIndex, MarkerIndex]; returns list [oJoint, mBody0, mBody1], containing the joint object number, and the two rigid body markers on body0/1 for the joint
\item[--]
{\bf example}: \vspace{-12pt}\ei\begin{lstlisting}[language=Python, xleftmargin=36pt]
  import exudyn as exu
  from exudyn.utilities import * #includes itemInterface and rigidBodyUtilities
  import numpy as np
  SC = exu.SystemContainer()
  mbs = SC.AddSystem()
  b0 = mbs.CreateRigidBody(inertia = InertiaCuboid(density=5000,
                                                   sideLengths=[1,0.1,0.1]),
                           referencePosition = [5,0,0],
                           initialAngularVelocity = [5,0,0],
                           gravity = [0,-9.81,0],
                           graphicsDataList = [exu.graphics.Brick(size=[1,0.1,0.1],
                                                                        color=exu.graphics.color.orange)])
  oGround = mbs.AddObject(ObjectGround())
  mbs.CreateSphericalJoint(bodyNumbers=[oGround, b0], position=[5.5,0,0],
                           useGlobalFrame=True, jointRadius=0.06)
  mbs.Assemble()
  simulationSettings = exu.SimulationSettings() #takes currently set values or default values
  simulationSettings.timeIntegration.numberOfSteps = 1000
  simulationSettings.timeIntegration.endTime = 2
  mbs.SolveDynamic(simulationSettings = simulationSettings)
\end{lstlisting}\vspace{-24pt}\bi\item[]\vspace{-24pt}\vspace{12pt}\end{itemize}
%

%
\noindent For examples on CreateSphericalJoint see Relevant Examples (Ex) and TestModels (TM) with weblink to github:
\bi
 \item \footnotesize \exuUrl{https://github.com/jgerstmayr/EXUDYN/blob/master/main/pythonDev/TestModels/driveTrainTest.py}{\texttt{driveTrainTest.py}} (TM), 
\exuUrl{https://github.com/jgerstmayr/EXUDYN/blob/master/main/pythonDev/TestModels/mainSystemExtensionsTests.py}{\texttt{mainSystemExtensionsTests.py}} (TM)
\ei

%
\begin{flushleft}
\noindent {def {\bf \exuUrl{https://github.com/jgerstmayr/EXUDYN/blob/master/main/pythonDev/exudyn/mainSystemExtensions.py\#L1219}{CreateGenericJoint}{}}}\label{sec:mainsystemextensions:CreateGenericJoint}
({\it name}= '', {\it bodyNumbers}= [None, None], {\it position}= [], {\it rotationMatrixAxes}= np.eye(3), {\it constrainedAxes}= [1,1,1, 1,1,1], {\it useGlobalFrame}= True, {\it offsetUserFunction}= 0, {\it offsetUserFunction\_t}= 0, {\it show}= True, {\it axesRadius}= 0.1, {\it axesLength}= 0.4, {\it color}= exudyn.graphics.color.default)
\end{flushleft}
\setlength{\itemindent}{0.7cm}
\begin{itemize}[leftmargin=0.7cm]
\item[--]
{\bf function description}: \vspace{-6pt}
\begin{itemize}[leftmargin=1.2cm]
\setlength{\itemindent}{-0.7cm}
\item[]Create generic joint between two bodies; definition of joint position (position) and axes (rotationMatrixAxes) in global coordinates (useGlobalFrame=True) or in local coordinates of body0 (useGlobalFrame=False), where rotationMatrixAxes is an additional rotation to body0; all markers, markerRotation and other quantities are automatically computed
\item[]- NOTE that this function is added to MainSystem via Python function MainSystemCreateGenericJoint.
\end{itemize}
\item[--]
{\bf input}: \vspace{-6pt}
\begin{itemize}[leftmargin=1.2cm]
\setlength{\itemindent}{-0.7cm}
\item[]{\it name}: name string for joint; markers get Marker0:name and Marker1:name
\item[]{\it bodyNumber0}: a object number for body0, must be rigid body or ground object
\item[]{\it bodyNumber1}: a object number for body1, must be rigid body or ground object
\item[]{\it position}: a 3D vector as list or np.array: if useGlobalFrame=True it describes the global position of the joint in reference configuration; else: local position in body0
\item[]{\it rotationMatrixAxes}: rotation matrix which defines orientation of constrainedAxes; if useGlobalFrame, this rotation matrix is global, else the rotation matrix is post-multiplied with the rotation of body0, identical with rotationMarker0 in the joint
\item[]{\it constrainedAxes}: flag, which determines which translation (0,1,2) and rotation (3,4,5) axes are constrained; each entry may only be 0 (=free) axis or 1 (=constrained axis); ALL constrained Axes are defined relative to reference rotation of body0 times rotation0
\item[]{\it useGlobalFrame}: if False, the position is defined in the local coordinate system of body0, otherwise it is defined in global coordinates
\item[]{\it offsetUserFunction}: a user function offsetUserFunction(mbs, t, itemNumber, offsetUserFunctionParameters)->float ; this function replaces the internal (constant) by a user-defined offset. This allows to realize rheonomic joints and allows kinematic simulation
\item[]{\it offsetUserFunction\_t}: a user function offsetUserFunction\_t(mbs, t, itemNumber, offsetUserFunctionParameters)->float ; this function replaces the internal (constant) by a user-defined offset velocity; this function is used instead of offsetUserFunction, if velocityLevel (index2) time integration
\item[]{\it show}: if True, connector visualization is drawn
\item[]{\it axesRadius}: radius of axes for connector graphical representation
\item[]{\it axesLength}: length of axes for connector graphical representation
\item[]{\it color}: color of connector
\end{itemize}
\item[--]
{\bf output}: [ObjectIndex, MarkerIndex, MarkerIndex]; returns list [oJoint, mBody0, mBody1], containing the joint object number, and the two rigid body markers on body0/1 for the joint
\item[--]
{\bf example}: \vspace{-12pt}\ei\begin{lstlisting}[language=Python, xleftmargin=36pt]
  import exudyn as exu
  from exudyn.utilities import * #includes itemInterface and rigidBodyUtilities
  import numpy as np
  SC = exu.SystemContainer()
  mbs = SC.AddSystem()
  b0 = mbs.CreateRigidBody(inertia = InertiaCuboid(density=5000,
                                                   sideLengths=[1,0.1,0.1]),
                           referencePosition = [6,0,0],
                           initialAngularVelocity = [0,8,0],
                           gravity = [0,-9.81,0],
                           graphicsDataList = [exu.graphics.Brick(size=[1,0.1,0.1],
                                                                        color=exu.graphics.color.orange)])
  oGround = mbs.AddObject(ObjectGround())
  mbs.CreateGenericJoint(bodyNumbers=[oGround, b0], position=[5.5,0,0],
                         constrainedAxes=[1,1,1, 1,0,0],
                         rotationMatrixAxes=RotationMatrixX(0.125*pi), #tilt axes
                         useGlobalFrame=True, axesRadius=0.02, axesLength=0.2)
  mbs.Assemble()
  simulationSettings = exu.SimulationSettings() #takes currently set values or default values
  simulationSettings.timeIntegration.numberOfSteps = 1000
  simulationSettings.timeIntegration.endTime = 2
  mbs.SolveDynamic(simulationSettings = simulationSettings)
\end{lstlisting}\vspace{-24pt}\bi\item[]\vspace{-24pt}\vspace{12pt}\end{itemize}
%

%
\noindent For examples on CreateGenericJoint see Relevant Examples (Ex) and TestModels (TM) with weblink to github:
\bi
 \item \footnotesize \exuUrl{https://github.com/jgerstmayr/EXUDYN/blob/master/main/pythonDev/Examples/bungeeJump.py}{\texttt{bungeeJump.py}} (Ex), 
\exuUrl{https://github.com/jgerstmayr/EXUDYN/blob/master/main/pythonDev/Examples/pistonEngine.py}{\texttt{pistonEngine.py}} (Ex), 
\exuUrl{https://github.com/jgerstmayr/EXUDYN/blob/master/main/pythonDev/Examples/universalJoint.py}{\texttt{universalJoint.py}} (Ex), 
\\ \exuUrl{https://github.com/jgerstmayr/EXUDYN/blob/master/main/pythonDev/TestModels/bricardMechanism.py}{\texttt{bricardMechanism.py}} (TM), 
\exuUrl{https://github.com/jgerstmayr/EXUDYN/blob/master/main/pythonDev/TestModels/complexEigenvaluesTest.py}{\texttt{complexEigenvaluesTest.py}} (TM), 
\exuUrl{https://github.com/jgerstmayr/EXUDYN/blob/master/main/pythonDev/TestModels/computeODE2AEeigenvaluesTest.py}{\texttt{computeODE2AEeigenvaluesTest.py}} (TM), 
\\ \exuUrl{https://github.com/jgerstmayr/EXUDYN/blob/master/main/pythonDev/TestModels/driveTrainTest.py}{\texttt{driveTrainTest.py}} (TM), 
\exuUrl{https://github.com/jgerstmayr/EXUDYN/blob/master/main/pythonDev/TestModels/generalContactImplicit2.py}{\texttt{generalContactImplicit2.py}} (TM), 
 ...

\ei

%
\begin{flushleft}
\noindent {def {\bf \exuUrl{https://github.com/jgerstmayr/EXUDYN/blob/master/main/pythonDev/exudyn/mainSystemExtensions.py\#L1332}{CreateDistanceConstraint}{}}}\label{sec:mainsystemextensions:CreateDistanceConstraint}
({\it name}= '', {\it bodyNumbers}= [None, None], {\it localPosition0}= [0.,0.,0.], {\it localPosition1}= [0.,0.,0.], {\it distance}= None, {\it bodyOrNodeList}= [None, None], {\it bodyList}= [None, None], {\it show}= True, {\it drawSize}= -1., {\it color}= exudyn.graphics.color.default)
\end{flushleft}
\setlength{\itemindent}{0.7cm}
\begin{itemize}[leftmargin=0.7cm]
\item[--]
{\bf function description}: \vspace{-6pt}
\begin{itemize}[leftmargin=1.2cm]
\setlength{\itemindent}{-0.7cm}
\item[]Create distance joint between two bodies; definition of joint positions in local coordinates of bodies or nodes; if distance=None, it is computed automatically from reference length; all markers are automatically computed
\item[]- NOTE that this function is added to MainSystem via Python function MainSystemCreateDistanceConstraint.
\end{itemize}
\item[--]
{\bf input}: \vspace{-6pt}
\begin{itemize}[leftmargin=1.2cm]
\setlength{\itemindent}{-0.7cm}
\item[]{\it name}: name string for joint; markers get Marker0:name and Marker1:name
\item[]{\it bodyNumbers}: a list of two body numbers (ObjectIndex) to be constrained
\item[]{\it localPosition0}: local position (as 3D list or numpy array) on body0, if not a node number
\item[]{\it localPosition1}: local position (as 3D list or numpy array) on body1, if not a node number
\item[]{\it distance}: if None, distance is computed from reference position of bodies or nodes; if not None, this distance is prescribed between the two positions; if distance = 0, it will create a SphericalJoint as this case is not possible with a DistanceConstraint
\item[]{\it bodyOrNodeList}: alternative to bodyNumbers; a list of object numbers (with specific localPosition0/1) or node numbers; may alse be mixed types; to use this case, set bodyNumbers = [None,None]
\item[]{\it show}: if True, connector visualization is drawn
\item[]{\it drawSize}: general drawing size of node
\item[]{\it color}: color of connector
\end{itemize}
\item[--]
{\bf output}: [ObjectIndex, MarkerIndex, MarkerIndex]; returns list [oJoint, mBody0, mBody1], containing the joint object number, and the two rigid body markers on body0/1 for the joint
\item[--]
{\bf example}: \vspace{-12pt}\ei\begin{lstlisting}[language=Python, xleftmargin=36pt]
  import exudyn as exu
  from exudyn.utilities import * #includes itemInterface and rigidBodyUtilities
  import numpy as np
  SC = exu.SystemContainer()
  mbs = SC.AddSystem()
  b0 = mbs.CreateRigidBody(inertia = InertiaCuboid(density=5000,
                                                    sideLengths=[1,0.1,0.1]),
                            referencePosition = [6,0,0],
                            gravity = [0,-9.81,0],
                            graphicsDataList = [exu.graphics.Brick(size=[1,0.1,0.1],
                                                                        color=exu.graphics.color.orange)])
  m1 = mbs.CreateMassPoint(referencePosition=[5.5,-1,0],
                           physicsMass=1, drawSize = 0.2)
  n1 = mbs.GetObject(m1)['nodeNumber']
  oGround = mbs.AddObject(ObjectGround())
  mbs.CreateDistanceConstraint(bodyNumbers=[oGround, b0],
                               localPosition0 = [6.5,1,0],
                               localPosition1 = [0.5,0,0],
                               distance=None, #automatically computed
                               drawSize=0.06)
  mbs.CreateDistanceConstraint(bodyOrNodeList=[b0, n1],
                               localPosition0 = [-0.5,0,0],
                               localPosition1 = [0.,0.,0.], #must be [0,0,0] for Node
                               distance=None, #automatically computed
                               drawSize=0.06)
  mbs.Assemble()
  simulationSettings = exu.SimulationSettings() #takes currently set values or default values
  simulationSettings.timeIntegration.numberOfSteps = 1000
  simulationSettings.timeIntegration.endTime = 2
  mbs.SolveDynamic(simulationSettings = simulationSettings)
\end{lstlisting}\vspace{-24pt}\bi\item[]\vspace{-24pt}\vspace{12pt}\end{itemize}
%

%
\noindent For examples on CreateDistanceConstraint see Relevant Examples (Ex) and TestModels (TM) with weblink to github:
\bi
 \item \footnotesize \exuUrl{https://github.com/jgerstmayr/EXUDYN/blob/master/main/pythonDev/Examples/chatGPTupdate.py}{\texttt{chatGPTupdate.py}} (Ex), 
\exuUrl{https://github.com/jgerstmayr/EXUDYN/blob/master/main/pythonDev/Examples/chatGPTupdate2.py}{\texttt{chatGPTupdate2.py}} (Ex), 
\exuUrl{https://github.com/jgerstmayr/EXUDYN/blob/master/main/pythonDev/TestModels/mainSystemExtensionsTests.py}{\texttt{mainSystemExtensionsTests.py}} (TM), 
\\ \exuUrl{https://github.com/jgerstmayr/EXUDYN/blob/master/main/pythonDev/TestModels/taskmanagerTest.py}{\texttt{taskmanagerTest.py}} (TM)
\ei

%
\begin{flushleft}
\noindent {def {\bf \exuUrl{https://github.com/jgerstmayr/EXUDYN/blob/master/main/pythonDev/exudyn/mainSystemExtensions.py\#L1453}{CreateForce}{}}}\label{sec:mainsystemextensions:CreateForce}
({\it name}= '', {\it bodyNumber}= None, {\it loadVector}= [0.,0.,0.], {\it localPosition}= [0.,0.,0.], {\it bodyFixed}= False, {\it loadVectorUserFunction}= 0, {\it show}= True)
\end{flushleft}
\setlength{\itemindent}{0.7cm}
\begin{itemize}[leftmargin=0.7cm]
\item[--]
{\bf function description}: \vspace{-6pt}
\begin{itemize}[leftmargin=1.2cm]
\setlength{\itemindent}{-0.7cm}
\item[]helper function to create force applied to given body
\item[]- NOTE that this function is added to MainSystem via Python function MainSystemCreateForce.
\end{itemize}
\item[--]
{\bf input}: \vspace{-6pt}
\begin{itemize}[leftmargin=1.2cm]
\setlength{\itemindent}{-0.7cm}
\item[]{\it name}: name string for object
\item[]{\it bodyNumber}: body number (ObjectIndex) at which the force is applied to
\item[]{\it loadVector}: force vector (as 3D list or numpy array)
\item[]{\it localPosition}: local position (as 3D list or numpy array) where force is applied
\item[]{\it bodyFixed}: if True, the force is corotated with the body; else, the force is global
\item[]{\it loadVectorUserFunction}: A Python function f(mbs, t, load)->loadVector which defines the time-dependent load and replaces loadVector in every time step; the arg load is the static loadVector
\item[]{\it show}: if True, load is drawn
\end{itemize}
\item[--]
{\bf output}: LoadIndex; returns load index
\item[--]
{\bf example}: \vspace{-12pt}\ei\begin{lstlisting}[language=Python, xleftmargin=36pt]
  import exudyn as exu
  from exudyn.utilities import * #includes itemInterface and rigidBodyUtilities
  import numpy as np
  SC = exu.SystemContainer()
  mbs = SC.AddSystem()
  b0=mbs.CreateMassPoint(referencePosition = [0,0,0],
                         initialVelocity = [2,5,0],
                         physicsMass = 1, gravity = [0,-9.81,0],
                         drawSize = 0.5, color=exu.graphics.color.blue)
  f0=mbs.CreateForce(bodyNumber=b0, loadVector=[100,0,0],
                     localPosition=[0,0,0])
  mbs.Assemble()
  simulationSettings = exu.SimulationSettings() #takes currently set values or default values
  simulationSettings.timeIntegration.numberOfSteps = 1000
  simulationSettings.timeIntegration.endTime = 2
  mbs.SolveDynamic(simulationSettings = simulationSettings)
\end{lstlisting}\vspace{-24pt}\bi\item[]\vspace{-24pt}\vspace{12pt}\end{itemize}
%

%
\noindent For examples on CreateForce see Relevant Examples (Ex) and TestModels (TM) with weblink to github:
\bi
 \item \footnotesize \exuUrl{https://github.com/jgerstmayr/EXUDYN/blob/master/main/pythonDev/Examples/cartesianSpringDamper.py}{\texttt{cartesianSpringDamper.py}} (Ex), 
\exuUrl{https://github.com/jgerstmayr/EXUDYN/blob/master/main/pythonDev/Examples/cartesianSpringDamperUserFunction.py}{\texttt{cartesianSpringDamperUserFunction.py}} (Ex), 
\exuUrl{https://github.com/jgerstmayr/EXUDYN/blob/master/main/pythonDev/Examples/chatGPTupdate.py}{\texttt{chatGPTupdate.py}} (Ex), 
\\ \exuUrl{https://github.com/jgerstmayr/EXUDYN/blob/master/main/pythonDev/Examples/chatGPTupdate2.py}{\texttt{chatGPTupdate2.py}} (Ex), 
\exuUrl{https://github.com/jgerstmayr/EXUDYN/blob/master/main/pythonDev/Examples/rigidBodyTutorial3.py}{\texttt{rigidBodyTutorial3.py}} (Ex), 
 ...
, 
\exuUrl{https://github.com/jgerstmayr/EXUDYN/blob/master/main/pythonDev/TestModels/mainSystemExtensionsTests.py}{\texttt{mainSystemExtensionsTests.py}} (TM), 
\\ \exuUrl{https://github.com/jgerstmayr/EXUDYN/blob/master/main/pythonDev/TestModels/taskmanagerTest.py}{\texttt{taskmanagerTest.py}} (TM)
\ei

%
\begin{flushleft}
\noindent {def {\bf \exuUrl{https://github.com/jgerstmayr/EXUDYN/blob/master/main/pythonDev/exudyn/mainSystemExtensions.py\#L1531}{CreateTorque}{}}}\label{sec:mainsystemextensions:CreateTorque}
({\it name}= '', {\it bodyNumber}= None, {\it loadVector}= [0.,0.,0.], {\it localPosition}= [0.,0.,0.], {\it bodyFixed}= False, {\it loadVectorUserFunction}= 0, {\it show}= True)
\end{flushleft}
\setlength{\itemindent}{0.7cm}
\begin{itemize}[leftmargin=0.7cm]
\item[--]
{\bf function description}: \vspace{-6pt}
\begin{itemize}[leftmargin=1.2cm]
\setlength{\itemindent}{-0.7cm}
\item[]helper function to create torque applied to given body
\item[]- NOTE that this function is added to MainSystem via Python function MainSystemCreateTorque.
\end{itemize}
\item[--]
{\bf input}: \vspace{-6pt}
\begin{itemize}[leftmargin=1.2cm]
\setlength{\itemindent}{-0.7cm}
\item[]{\it name}: name string for object
\item[]{\it bodyNumber}: body number (ObjectIndex) at which the torque is applied to
\item[]{\it loadVector}: torque vector (as 3D list or numpy array)
\item[]{\it localPosition}: local position (as 3D list or numpy array) where torque is applied
\item[]{\it bodyFixed}: if True, the torque is corotated with the body; else, the torque is global
\item[]{\it loadVectorUserFunction}: A Python function f(mbs, t, load)->loadVector which defines the time-dependent load and replaces loadVector in every time step; the arg load is the static loadVector
\item[]{\it show}: if True, load is drawn
\end{itemize}
\item[--]
{\bf output}: LoadIndex; returns load index
\item[--]
{\bf example}: \vspace{-12pt}\ei\begin{lstlisting}[language=Python, xleftmargin=36pt]
  import exudyn as exu
  from exudyn.utilities import * #includes itemInterface and rigidBodyUtilities
  import numpy as np
  SC = exu.SystemContainer()
  mbs = SC.AddSystem()
  b0 = mbs.CreateRigidBody(inertia = InertiaCuboid(density=5000,
                                                   sideLengths=[1,0.1,0.1]),
                           referencePosition = [1,3,0],
                           gravity = [0,-9.81,0],
                           graphicsDataList = [exu.graphics.Brick(size=[1,0.1,0.1],
                                                                        color=exu.graphics.color.red)])
  f0=mbs.CreateTorque(bodyNumber=b0, loadVector=[0,100,0])
  mbs.Assemble()
  simulationSettings = exu.SimulationSettings() #takes currently set values or default values
  simulationSettings.timeIntegration.numberOfSteps = 1000
  simulationSettings.timeIntegration.endTime = 2
  mbs.SolveDynamic(simulationSettings = simulationSettings)
\end{lstlisting}\vspace{-24pt}\bi\item[]\vspace{-24pt}\vspace{12pt}\end{itemize}
%

%
\noindent For examples on CreateTorque see Relevant Examples (Ex) and TestModels (TM) with weblink to github:
\bi
 \item \footnotesize \exuUrl{https://github.com/jgerstmayr/EXUDYN/blob/master/main/pythonDev/Examples/chatGPTupdate.py}{\texttt{chatGPTupdate.py}} (Ex), 
\exuUrl{https://github.com/jgerstmayr/EXUDYN/blob/master/main/pythonDev/Examples/chatGPTupdate2.py}{\texttt{chatGPTupdate2.py}} (Ex), 
\exuUrl{https://github.com/jgerstmayr/EXUDYN/blob/master/main/pythonDev/Examples/rigidBodyTutorial3.py}{\texttt{rigidBodyTutorial3.py}} (Ex), 
\\ \exuUrl{https://github.com/jgerstmayr/EXUDYN/blob/master/main/pythonDev/TestModels/mainSystemExtensionsTests.py}{\texttt{mainSystemExtensionsTests.py}} (TM)
\ei

%
